\textbf{Resumen}:
Desde el auge de la navegación autónoma en los últimos años, esta ha empezado a tener un papel crucial en diversas industrias y aplicaciones, desde la logística y la agricultura hasta la exploración espacial. El dominio de la navegación autónoma se ha convertido en un aspecto fundamental para garantizar el éxito y la eficiencia de estas tecnologías.

En el presente trabajo se aborda el desafío de permitir que un robot móvil dotado de un sistema sensorial navegue de manera autónoma en el exterior. Se han integrado y desarrollado componentes de software para conseguir una localización y una navegación precisas junto con un sistema de control y monitorización. Para ello, se han explorado diversas técnicas de localización y navegación. También se ha investigado en profundidad sobre el hardware de los sensores y la manera de interconectarlos a ROS2 mediante drivers, para el control adaptativo y robusto del trayecto.

Aunque el sistema también se ha probado en simulación, principalmente ha sido testado en los entornos de la Facultad de Telecomunicaciones de la Universidad de Málaga.

Los resultados han mostrado una buena implementación de las herramientas disponibles, junto con un sistema robusto y fiable de navegación autónoma en exteriores.

\textbf{Palabras claves}: Robótica Móvil, ROS2, Navegación Autónoma, LIDAR, GPS RTK, Localización, EKF, Planificadores, árboles de comportamiento. 

\vspace{1cm}
\begin{center}
  \rule{0.5\textwidth}{.4pt}
\end{center}
\vspace{1cm}

\textbf{Abstract}:Since the rise of autonomous navigation in recent years, it has begun to play a crucial role in various industries and applications, from logistics and agriculture to space exploration. Mastery of autonomous navigation has become a fundamental aspect to ensure the success and efficiency of these technologies.

In this work, the challenge of enabling a mobile robot equipped with a sensory system to navigate autonomously outdoors is addressed. Software components have been integrated and developed to achieve precise localization and navigation, along with a control and monitoring system. To this end, various localization and navigation techniques have been explored. Additionally, an in-depth investigation into the sensor hardware and the way to interconnect them to ROS2 via drivers has been conducted, for the adaptive and robust control of the path.

Although the system has also been tested in simulation, it has mainly been tested in the environments of the Faculty of Telecommunications at the University of Málaga.

The results have shown a good implementation of the available tools, along with a robust and reliable outdoor autonomous navigation system.

\textbf{Keywords}: Mobile Robotics, ROS2, Autonomous Navigation, LIDAR, GPS RTK, Localization, EKF, Planners, behavior trees.
