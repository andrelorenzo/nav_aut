\textbf{Resumen}:
Desde el auge de la navegación autónoma en los últimos años, está ha empezado a tener un papel crucial en diversas industrias y 
aplicaciones, desde la logística y la agricultura hasta la exploración espacial, el dominio de la navegación autónoma se ha convertido 
en un aspecto fundamental para garantizar el éxito y la eficiencia de estas tecnologías.

En el presente trabajo se aborda el desafió de permitir que un robot móvil de uso investigacional dotado de un sistema sensorial, 
navegue de manera autónoma en el exterior. 
Se han integrado y desarrollado componentes de software para conseguir una localización y una navegación precisa junto con un sistema de 
control y monitorización, para ello se han explorado diversas técnica de localización y navegación, también se ha investigado en profundidad 
sobre el hardware de los sensores y la manera de interconectados a ROS2 mediante \textit{drivers}, para el control adaptativo y robusto del trayecto.


EL sistema aunque también se ha probado en simulación, principalmente ha sido testado en los entornos de la facultad de Telecomunicaciones 
de la Universidad de Málaga.

Los resultados han dejado ver una buena implementación de las herramientas a disposición junto con un sistema robusto y fiable de navegación 
autónoma en exteriores.

\textbf{Palabras claves}: Robótica Móvil, ROS2, Navegación Autónoma, LIDAR, GPS RTK, Localización, EKF, Planificadores, árboles de comportamiento. 

\vspace{1cm}
\begin{center}
  \rule{0.5\textwidth}{.4pt}
\end{center}
\vspace{1cm}

\textbf{Abstract}:Since the rise of autonomous navigation in recent years, it has begun to play a crucial role in various industries and applications, from logistics and agriculture to space exploration. The mastery of autonomous navigation has become a fundamental aspect in ensuring the success and efficiency of these technologies.

This paper addresses the challenge of enabling an investigational mobile robot equipped with a sensory system to navigate autonomously outdoors. Software components have been integrated and developed to achieve precise localization and navigation, along with a control and monitoring system. Various techniques for localization and navigation have been explored, as well as in-depth research on sensor hardware and their interconnection to ROS2 via drivers, for adaptive and robust path control.

Although the system has also been tested in simulation, it has primarily been tested in the environments of the Telecommunications Faculty of the University of Malaga.

The results have demonstrated a good implementation of the available tools along with a robust and reliable system for autonomous outdoor navigation.

\textbf{Keywords}: Mobile Robotics, ROS2, Autonomous Navigation, LIDAR, GPS RTK, Localization, EKF, Planners, behavior trees.
