\textbf{Resumen}:
La navegación autónoma de robots ha cobrado gran relevancia en los últimos años y promete tener un impacto significativo en el futuro de diversas industrias, como la logística, la agricultura, o la exploración espacial. Este proyecto aborda el desafio de dotar a un robot móvil de tan valiosa capacidad, navegar de forma autónoma, particularizandolo para la navegación en exteriores a traves de una secuencia de puntos de control que debe visitar. 

Se aborda tanto el diseño y configuración del sistema sensorial, el cual  incluye GPS+RTK, IMU, Lidar 3D y odometría, como el estudio, configuración  y adaptación de algoritmos de localización, detección y evitación de obstáculos., así como de navegación autónoma. El objetivo de la integración de todos estos componenetes   es lograr una navegación precisa, fiable y robusta que posibilite la realización de tareas de alto nivel. 

Un extenso abanico de pruebas reales con un robot móvil en configuración Ackerman, validan el corecto diseño y configuración de los módulos implicados en la navegación autónoma.

\textbf{Palabras claves}: Robótica Móvil, ROS2, Navegación Autónoma, LIDAR, GPS RTK, Localización, EKF, Planificadores, árboles de comportamiento. 

\vspace{1cm}
\begin{center}
  \rule{0.5\textwidth}{.4pt}
\end{center}
\vspace{1cm}

\textbf{Abstract}:Autonomous robot navigation has gained significant relevance in recent years and promises to have a substantial impact on the future of various industries, such as logistics, agriculture, and space exploration. This project addresses the challenge of endowing a mobile robot with such a valuable capability: autonomous navigation, specifically tailored for outdoor navigation through a sequence of control points it must visit.

The project covers both the design and configuration of the sensory system, which includes GPS+RTK, IMU, 3D Lidar, and odometry, as well as the study, configuration, and adaptation of localization, obstacle detection and avoidance, and autonomous navigation algorithms. The goal of integrating all these components is to achieve precise, reliable, and robust navigation that enables the execution of high-level tasks.

An extensive array of real-world tests with a mobile robot in Ackerman configuration validates the correct design and configuration of the modules involved in autonomous navigation.

\textbf{Keywords}: Mobile Robotics, ROS2, Autonomous Navigation, LIDAR, GPS RTK, Localization, EKF, Planners, behavior trees.
